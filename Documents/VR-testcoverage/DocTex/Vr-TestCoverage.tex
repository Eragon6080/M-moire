\documentclass[a4paper,10pt, oneside]{article}
%package

\usepackage[utf8]{inputenc}
\usepackage[T1]{fontenc}
\usepackage[french]{babel}
\usepackage{cite}
\usepackage{url}
\usepackage{pdfpages}
\usepackage{soul}
\usepackage{color}
\usepackage{tcolorbox}
\usepackage{graphicx}
\usepackage{geometry}
 \geometry{textwidth=17cm,textheight=26cm}
\newcommand{\bbox}{\begin{tcolorbox}[colback=red!5!white,
	colframe=red!75!black]\begin{center}}
\newcommand{\ebox}{\end{center}\end{tcolorbox}}
\newcommand{\red}{\colorbox{red}}
\newcommand{\blue}{\colorbox{blue}}
\newcommand{\yellow}{\colorbox{yellow}}
\newcommand{\myit}{\textit}
\newcommand{\mybf}{\textbf}
\newcommand{\li}{\newline}

\graphicspath{{./Images/}}


\geometry{textwidth=16cm, textheight=26cm}

\title{Fiche de lecture : VR-TestCoverage : Test Coverage Vizualization and immersion in Virtual Reality}

\author{Matthys Gaillard}

\date{\today}

\begin{document}
\maketitle
\section{\ul{Pourquoi ce choix ?}}
        \par J'ai choisi ce document\cite{A1}, car il proposait une autre manière de visualiser un programme. Au lieu de voir
        directement la structure du code en analysant directement les fichiers sources, il propose d'analyser directement les tests.
        De fait, cela changeait radicalement des autres documents trouvés précédemment.
\section{\ul{Analyse du document}}
\subsection{\ul{Contexte}}
        \par Les chercheurs ont remarqué que les développeur faisaient souvent face à un dilemne : est-ce que leurs tests sont suffisamment efficaces? 
        Cela devenait aussi de plus en plus difficile de comprendre le code et les tests au fur et à mesure que les projets grossissaient. Ils pensaient donc, 
        qu'avec un environnement 3D interactif, ils pourraient amener une compréhension plus aisée et efficace de ces 2 aspects.
\subsection{\ul{Objectifs}}
        \par Proposer un environnement 3D pour visualiser les tests de couvertures.
\subsection{\ul{Méthode}}
        \par Ils ont utilisé un environnement 3D où ils ont utilisé un tree map utilisant le paradigme de la pyramide pour empiler la représentation des fichiers.
        Pour cela, ils ont utilisé C\# et Unity pour créer l'environnement 3D. Ils ont aussi utilisé 2 plugins propres à ce langage pour analyser les tests de couvertures.
        Ce sont les résultats de ces 2 plugins qui permettaient de collecter les données de tests et de savoir si ceux-ci réussissaient. Au final, les résultats des tests étaient retranscrits dans un format JSON spécifique permettant 
        facilement la lecture par un autre système quelconque. \li 
        \par Ils ont basé leur travail sur 3 tests : le test de couverture, le test de résultat et le test des dépendances.
\subsection{\ul{Résultats}}
        \par Pour ce travail, ils en conclu que la VR peut aider les développeurs à déterminer les zones de leur code qui ont été négligées ou moins y faire attention en vue d'une future correction.
\newpage

\bibliographystyle{apalike-fr}
\bibliography{biblio}
\end{document}
