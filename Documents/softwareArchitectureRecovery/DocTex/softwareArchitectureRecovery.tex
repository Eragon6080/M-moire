\documentclass[a4paper,10pt, oneside]{article}
%package

\usepackage[utf8]{inputenc}
\usepackage[T1]{fontenc}
\usepackage[french]{babel}
\usepackage{cite}
\usepackage{url}
\usepackage{pdfpages}
\usepackage{soul}
\usepackage{color}
\usepackage{tcolorbox}
\usepackage{graphicx}
\usepackage{geometry}

\newcommand{\bbox}{\begin{tcolorbox}[colback=red!5!white,
	colframe=red!75!black]\begin{center}}
\newcommand{\ebox}{\end{center}\end{tcolorbox}}
\newcommand{\red}{\colorbox{red}}
\newcommand{\blue}{\colorbox{blue}}
\newcommand{\yellow}{\colorbox{yellow}}
\newcommand{\myit}{\textit}
\newcommand{\mybf}{\textbf}
\newcommand{\li}{\newline}

\geometry{textwidth=16cm, textheight=26cm}

\title{Fiche de lecture : Utilizing Software Architecture Recovery to Explore Large-Scale Software Systems in Virtual Reality}

\author{Matthys Gaillard}

\date{\today}

\begin{document}
\maketitle
\section{\ul{Pourquoi ce choix ?}}
    \par J'ai choisi ce document\cite{A1}, car en lisant l'abstract et l'introduction, je me suis rendu compte qu'il pouvait être très complet.
    De plus, dès que j'avais fini de survoler la structure du document, j'ai pu m'affirmer effectivement de la qualité du document.
    Contrairement à d'autres documents, il explicite très clairement la méthode scientifique utilisée pour ses différentes tests en apportant
    à chaque fois les détails nécessaires pour comprendre les résultats.\li
    \par De plus, l'article détaille aussi les détails de ce qu'il cherche à prouver et à partager durant l'expérience.
\section{\ul{Analyse du document}}
\subsection{\ul{Contexte}}
        \par Les chercheurs à l'origine de cette article ont remarqué qu'il était bien compliqué de comprendre l'architecture d'un système logiciel non familier,
        surtout si une personne ne se base uniquement sur la compréhension directe du code. Cependant, ils expliquent quand même qu'il existe des outils permettant 
        de visualiser un système en privilégiant la structure directe du code (découpe en package, etc) et en négligeant la structure indirect de celui-ci.\li
        \par Par cela, je veux dire la dépendance des classes entre elles, mais aussi la taille du code, etc. Ces choses-là sont compliquées à appréhender
        avec des outils classiques, car ils ne permettent pas de visualiser l'ensemble du système en même temps. 
\subsection{\ul{Objectifs}}
        \par Ils ont décidé de créer une approche automatique pour la récupération de l'architecture logicielle et utiliser ses résultats dans un logicielle d'immersion 3D
        pour enfin accéder à ses connaissances cachées.
\subsection{\ul{Méthode}}
        \par Ils ont fait appel à 54 participants de l'université IT de Copenhague. Ils ont été répartis en 3 groupes de 18 personnes. Chaque groupe a été soumis à une méthode.ù
        Ils ont équilibré les équipes en fonction des connaissances pour éviter de trop grande disparité.\li
        \par Le premier groupe a été soumis à une méthode classique de visualisation de code via un éditeur de code (Eclipse). Le second groupe a été soumis à une méthode de visualisation de code en 3D.\li
        \par L'expérience était divisée en 3 parties : 
        \begin{enumerate}
            \item Une enquête sur les expériences des participants sur la VR et en programmation.
            \item Entraînement sur l'équipement qui sera utilisé par la suite.
            \item La phase de test (Eclipse ou VR).
        \end{enumerate}
        \par Après l'expérience, chaque participant a dû répondre à un questionnaire sur l'expérience vécue.
\subsection{\ul{Résultats}}
        \par Les résultats montrent que leur approche est plus efficace pour fournir un accès plus facile à l'information et donc, une meilleure compréhension
        du système cible et de son architecture et ainsi les relations entre les différents éléments le composant.

\bibliographystyle{plain}
\bibliography{biblio} 
\end{document}