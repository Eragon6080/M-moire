\documentclass[a4paper,10pt, oneside]{article}
%package

\usepackage[utf8]{inputenc}
\usepackage[T1]{fontenc}
\usepackage[french]{babel}
\usepackage{cite}
\usepackage{url}
\usepackage{pdfpages}
\usepackage{soul}
\usepackage{color}
\usepackage{tcolorbox}
\usepackage{graphicx}
\usepackage{geometry}

\newcommand{\bbox}{\begin{tcolorbox}[colback=red!5!white,
	colframe=red!75!black]\begin{center}}
\newcommand{\ebox}{\end{center}\end{tcolorbox}}
\newcommand{\red}{\colorbox{red}}
\newcommand{\blue}{\colorbox{blue}}
\newcommand{\yellow}{\colorbox{yellow}}
\newcommand{\myit}{\textit}
\newcommand{\mybf}{\textbf}
\newcommand{\li}{\newline}

\geometry{textwidth=16cm, textheight=26cm}

\title{Fiche de lecture : Collaborative software visualization for program comprehension}

\author{Matthys Gaillard}

\date{\today}

\begin{document}
\maketitle
\section{\ul{Pourquoi ce choix ?}}
		\par J'ai choisi ce document\cite{A1}, car il aborde un autre côté de la compréhension de code. Il parle ici de la comprehension de code via la collaboration,
		un peu à la manière de la "pair programming". Il est intéressant de voir comment la collaboration peut aider à la compréhension de code.\li
		\par De plus, au lieu de simplement d'être simplement assis, l'un à côté de l'autre, les auteurs ont développé un logiciel 
		permettant de collaborer à distance et à travers des outils différents, mais à travers un même logiciel tournant sur un navigateur web.
		On retrouve notamment comme appareil : un ordinateur, une tablette (Ipad pour la réalité augmentée) et un casque de réalité augmentée.

\section{\ul{Analyse du document}}
\subsection{\ul{Contexte}}
		\par Les chercheurs ont remarqué que dans le contexte de la compréhension de code, la collaboration est un facteur important. Cependant,
		peu d'étude ont été mené sur la collaboration de code dans la visualisation de code et surtout également dans la conception de logiciel par pairs.\li 
		\par Dès lors, les auteurs ont crée un logiciel permettant de visualiser un projet de manière collaborative en utilisant des appareils différents. En effet, la visualisation de code est encore
		à l'heure actuelle une tâche bien souvent menée seul ou alors elle est portée sur un seul type d'appareil à la fois.
\subsection{\ul{Objectifs}}
		\par L'objectif de cette étude est de voir si la collaboration dans la visualisation de code à travers la méthode de la ville est efficace à partir d'appareils différents.
		En effet, dans cette optique-là, il cherche à mesurer l'utilité et la joie procuré par la collaboration peu importe l'appareil utilisé.
\subsection{\ul{Méthode}}
		\par Comme dit plus haut, le logiciel est prévu pour tourner peu importe l'appareil employé. Pour cela, ils ont crée leur logiciel en l'hébergeant sur le cloud et ont fait
		en sorte qu'il tourne via un navigateur web en utilisant Three.js.\li
		\par Pour tester leur logiciel, ils ont fait appel à 20 étudiants : 19 étaient en bachelier d'informatique et 1 en master. Les 3 types d'appareil étaient utilisés en même temps.\li
		\par Pour quantifier les résultats, ils ont demandé aux étudiants de remplir un questionnaire pré-étude, un à chaque fois qu'ils utilisaient un appareil et un après l'étude pour un feedback final.
		Pendant toute la durée de l'expérience, il y avait en permanence 2 chercheurs qui aidaient et fournissaient des explications si nécessaires.
\subsection{\ul{Résultats}}
		\par Les résultats sont assez concluants. En effet, la majorité des candidats ont trouvé l'approche convaincante et que la collaboration est une bonne manière de comprendre le code.
\newpage
\bibliographystyle{apalike-fr}
\bibliography{biblio} 
\end{document}