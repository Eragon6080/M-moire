\documentclass[a4paper,10pt, oneside]{article}
%package

\usepackage[utf8]{inputenc}
\usepackage[T1]{fontenc}
\usepackage[french]{babel}
\usepackage{cite}
\usepackage{url}
\usepackage{pdfpages}
\usepackage{soul}
\usepackage{color}
\usepackage{tcolorbox}
\usepackage{graphicx}
\usepackage{geometry}
 \geometry{textwidth=17cm,textheight=26cm}
\newcommand{\bbox}{\begin{tcolorbox}[colback=red!5!white,
	colframe=red!75!black]\begin{center}}
\newcommand{\ebox}{\end{center}\end{tcolorbox}}
\newcommand{\red}{\colorbox{red}}
\newcommand{\blue}{\colorbox{blue}}
\newcommand{\yellow}{\colorbox{yellow}}
\newcommand{\myit}{\textit}
\newcommand{\mybf}{\textbf}
\newcommand{\li}{\newline}

\graphicspath{{./Images/}}


\geometry{textwidth=16cm, textheight=26cm}

\title{Fiche de lecture : Collaborative, Code-Proximal Dynamic Software Visualization within Code Editors}

\author{Matthys Gaillard}

\date{\today}

\begin{document}
\maketitle
\section{\ul{Pourquoi ce choix ?}}  
    \par J'ai choisi ce document\cite{A32}, car il propose une visualisation de groupe. Même si lors de cette expérimentation, il n'utilisait pas la réalité virtuelle,
    il proposait une solution par laquelle on pouvait visualiser le code en direct à travers une visualisation. De plus, l'extension développée pour Visual Studio Code
    permettait de collaborer en équipe et de pouvoir interagir avec elle.
\section{\ul{Analyse du document}}
\subsection{\ul{Contexte}}
    \par Les chercheurs ont remarqué que souvent les outils de visualisation sont des outils s'utilisant seul et permettait seulement de voir la structure du code, mais seulement
    de manière statique. De plus, ce sont souvent des outils externes aux IDE, forçant les utilisateurs à changer de permanence d'environnement et de plus, de relancer l'outil à chaque modification.\li
    \par Les chercheurs ont donc voulu créer un outil de visualisation dynamique, permettant de voir les changements en temps réel et de manière collaborative. Ils ont donc créé une extension pour Visual Studio Code, permettant de visualiser le code en temps réel et de manière collaborative.
\subsection{\ul{Objectifs}}
    \par L'objectif de cette étude est de montrer via une extension de Visual Studio Code, qu'il est possiblke de visualiser du code en temps réel tout en le modifiant.
    Leur métaphore de visualisation est celle de la ville. 
               
\subsection{\ul{Méthode}}
\subsection{\ul{Résultats}}
\newpage
\bibliographystyle{apalike-fr}
\bibliography{biblio}
\end{document}