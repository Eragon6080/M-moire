\documentclass[a4paper,10pt, oneside]{article}
%package

\usepackage[utf8]{inputenc}
\usepackage[T1]{fontenc}
\usepackage[french]{babel}
\usepackage{cite}
\usepackage{url}
\usepackage{pdfpages}
\usepackage{soul}
\usepackage{color}
\usepackage{tcolorbox}
\usepackage{graphicx}
\usepackage{geometry}
 \geometry{textwidth=17cm,textheight=26cm}
\newcommand{\bbox}{\begin{tcolorbox}[colback=red!5!white,
	colframe=red!75!black]\begin{center}}
\newcommand{\ebox}{\end{center}\end{tcolorbox}}
\newcommand{\red}{\colorbox{red}}
\newcommand{\blue}{\colorbox{blue}}
\newcommand{\yellow}{\colorbox{yellow}}
\newcommand{\myit}{\textit}
\newcommand{\mybf}{\textbf}
\newcommand{\li}{\newline}

\graphicspath{{./Images/}}


\geometry{textwidth=16cm, textheight=26cm}

\title{Fiche de lecture : Data Visceralization}

\author{Matthys Gaillard}

\date{\today}

\begin{document}
\maketitle
\section{\ul{Pourquoi ce choix ?}}
	\par La raison pour laquelle j'ai choisi ce document\cite{A5} est très simple. Le sujet de l'article n'était pas directement axé sur la visualisation de données, mais plutôt 
	sur un dérivé que les chercheurs appellent ici la data visceralization. Ce terme m'a intrigué et j'ai donc décidé de lire l'article pour en savoir plus. Et après lecture,
	je me suis rendu compte que ce sujet était une autre manière de visualiser des données, autrement que la visualisation à l'aide de codeCity.

\section{\ul{Analyse du document}}
\subsection{\ul{Contexte}} 
		\par Une grande partie de la visualisation de données consiste à transformer des données brutes vers des données visuelles bien plus parlantes pour tout le monde.
		Tout cela repose sur le principe de l'abstraction, ce qui permet d'ajouter une couche de sens supplémentaire aux données. La réalité virtuelle permet de la rendre encore
		encore plus naturelle. Cette manière de visualiser les données peuvent aider à la compréhension des métriques de base dans l'abstraction de données.
\subsection{\ul{Objectifs}}
		\par Les objectifs de cette recherche sont de trouver une manière de visualiser les données de manière plus naturelle, plus proche de la réalité. Pour cela, les chercheurs
		ont décidé de se baser sur la réalité virtuelle, qui permet de rendre les choses plus naturelles. Ils ont donc décidé de créer un environnement virtuel dans lequel les
		données seraient représentées par des objets. Ces objets seraient ensuite manipulables par l'utilisateur, ce qui permettrait de mieux comprendre les données.
\subsection{\ul{Méthode}}
		\par Pour tester leurs recherches, les chercheurs ont décidé de créer plusieurs prototypes pour essayer de voir s'ils aidaient plus facilement à comprendre les données.
		Ils ont également fait appel à des intervenants externes.
\subsection{\ul{Résultats}}
\par La data visceralization et donc, l'utilisation de la réalité virtuelle pour visualiser les données, permet de mieux comprendre les données et cela de manière plus intuitive.
\newpage

\bibliographystyle{plain}
\bibliography{biblio}
\end{document}