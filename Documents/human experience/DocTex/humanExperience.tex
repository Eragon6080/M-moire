\documentclass[a4paper,10pt, oneside]{article}
%package

\usepackage[utf8]{inputenc}
\usepackage[T1]{fontenc}
\usepackage[french]{babel}
\usepackage{cite}
\usepackage{url}
\usepackage{pdfpages}
\usepackage{soul}
\usepackage{color}
\usepackage{tcolorbox}
\usepackage{graphicx}
\usepackage{geometry}

\newcommand{\bbox}{\begin{tcolorbox}[colback=red!5!white,
	colframe=red!75!black]\begin{center}}
\newcommand{\ebox}{\end{center}\end{tcolorbox}}
\newcommand{\red}{\colorbox{red}}
\newcommand{\blue}{\colorbox{blue}}
\newcommand{\yellow}{\colorbox{yellow}}
\newcommand{\myit}{\textit}
\newcommand{\mybf}{\textbf}
\newcommand{\li}{\newline}

\geometry{textwidth=16cm, textheight=26cm}

\title{Fiche de lecture : The human experience of comprehending source code in virtual reality}

\author{Matthys Gaillard}

\date{\today}

\begin{document}
\maketitle
\section{\ul{Pourquoi ce choix ?}}
		\par J'ai choisi ce document\cite{A10}, car il ne se repose pas uniquement sur une manière de comprendre du code via la réalité virtuelle.
		En effet, celui-ci se concentre plus sur l'aspect physique de la compréhension du code, et non sur l'aspect virtuel. Donc, si je peux résumé cela,
		je dirai qu'est ce qui se passe dans notre corps quand on lit du code. Il se base sur l'expérience humaine de la compréhension du code.
\section{\ul{Analyse du document}}
\subsection{\ul{Contexte}}
		\par Les auteurs ont réalises que beaucoup de domaines ont commencé à utilisé la réalité virtuelle pour le travail comme la psychologie, la médecine, l'architecture et les jeux-vidéos.
		D'autres études plus récentes tentent à montrer que la VR peut être utilisé comme un outil par les développeurs. Cependant, ils ont remarqué que très peu d'études sur le sujet s'intéressent à la
		compréhension de code à l'aide de l'outil.
\subsection{\ul{Objectifs}}
		\par L'objectif de cette étude n'est pas vraiment la compréhension du code en tant que tel, mais plutôt de voir l'aspect humain de cette compréhension\footnote{Les auteurs de l'article appelle cela l'expérience humaine}.
		Parmi ces facteurs, on retrouve la concentration, la charge mental (fatigue), la frustration, la charge physique et la productivité.
\subsection{\ul{Méthode}}
		\par Pour cela, ils ont fait appel à 26 étudiants, tous diplômé en informatique. Ils devaient remplir un formulaire avant de commencer l'expérience. Durant l'expérience, chacun
		devait écrire une sorte de résumé de ce qu'ils avaient compris du code. Ensuite, ils répondaient de nouveau à un questionnaire. Pendant, l'expérience, il récoltait différentes métriques sur l'activité perçue du cerveau.
		Á partir de là, à l'aide de statistiques, ils ont pu déterminer les facteurs qui influençaient la compréhension du code.\li
		\par L'expérience se menait de sorte à ce qu'ils aient 2 groupes de 13 personnes : un groupe menait l'expérience sur la réalité virtuelle et l'autre sur un ordinateur personnel classique.
		En ce qui concernait le code en lui-même, il devait analyser 8 projets écrits en Java.
\subsection{\ul{Résultats}}
		\par Les résultats de l'expérience concluent que la réalité virtuelle est de affecte négativement les capacités cognitives dans la compréhension de code que sur un environnement plus classique.
		Cependant, il n'y a pas de différence significative quant à la productivité. En effet, les participants ont réussi à comprendre le code de la même manière dans les deux environnements.
\newpage
\bibliographystyle{apalike-fr}
\bibliography{biblio} 
\end{document}
    
