\documentclass[a4paper,10pt, oneside]{article}
%package

\usepackage[utf8]{inputenc}
\usepackage[T1]{fontenc}
\usepackage[french]{babel}
\usepackage{cite}
\usepackage{url}
\usepackage{pdfpages}
\usepackage{soul}
\usepackage{color}
\usepackage{tcolorbox}
\usepackage{graphicx}
\usepackage{geometry}

\geometry{textwidth=16cm, textheight=26cm}

\title{Fiche de lecture : Code Park : A New 3D Code Visualization Tool}

\author{Matthys Gaillard}

\date{\today}

\begin{document}
\maketitle
\section{\ul{Pourquoi ce choix ?}}
    \par J'ai choisi ce document\cite{A0} pour la simple raison qu'il propose une autre méthode pour visualiser la structure du code d'un programme. 
    En effet, dans cette étude, il propose de visualiser la structure du code si on était dans un parc où l'on peut se promener librement.
    \par De plus, il propose une comparaison avec Code City, un autre logiciel de visualisation de code en 3D.
\section{\ul{Analyse du document}}
\subsection{\ul{Contexte}}
    \par Les auteurs ont remarqué que les techniques de visualisation de code existantes ne permettaient pas de visualiser la structure du code de manière efficace.
    Différentes techniques sont apparues au fur et à mesure des années. Ils citent notamment SeeSoft, CodeBubble ou encore CodeCity.
    \par Cependant, pour eux, peu d'attention ont été porté sur la visualisation pure du code
\subsection{\ul{Objectifs}}
    \par Leur but est donc de développer CodePark, c'est une application au moment de l'écriture de l'article de ne visualiser que du code écrit en C\#.
    Ils veulent que l'application soient plus engageantes et plus amusantes vis à vis de l'apprentissage de la structure du code. Pour aider à la mémorisation, ils comptent 
    sur les capacités mémorielles se basant sur la perception spatiale.
\subsection{\ul{Méthode}}
    \par Pour leur expérience, les chercheurs ont décidé de créer 2 groupe :
    \begin{enumerate}
        \item Un groupe avec CodePark : réalité virtuelle.
        \item Un groupe avec Visual Studio
    \end{enumerate}
    \par Ils ont demandé de l'aide à 28 participants de les aider dans l'étude. Les seuls prérequis demandés était une connaissance préalable du C\#. Ils ont du participé à un pré-questionnaire visant 
    à savoir s'ils connaissaient effectivement le langage et un autre, post-étude servait à savoir leur ressenti vis à vis de l'expérience.
\subsection{\ul{Résultats}}
    \par L'utilisation de Code Park est un atout pour la compréhension d'une structure de code surtout grâce à l'accent mis sur l'expérience utilisateur.
    L'atout de la 3D et l'utilisation de la perception spatiale de l'être humain a également été démontrée. Les utilisateurs utilisant le casque de réalité virtuelle effectuaient 
    plus rapidement leurs tâches que ceux attablés devant leur ordinateur.
\newpage
\bibliographystyle{plain}
\bibliography{biblio}
\end{document}