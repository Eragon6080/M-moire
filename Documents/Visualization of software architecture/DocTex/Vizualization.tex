\documentclass[a4paper,10pt, oneside]{article}
%package

\usepackage[utf8]{inputenc}
\usepackage[T1]{fontenc}
\usepackage[french]{babel}
\usepackage{cite}
\usepackage{url}
\usepackage{pdfpages}
\usepackage{soul}
\usepackage{color}
\usepackage{tcolorbox}
\usepackage{graphicx}
\usepackage{geometry}
 \geometry{textwidth=17cm,textheight=26cm}
\newcommand{\bbox}{\begin{tcolorbox}[colback=red!5!white,
	colframe=red!75!black]\begin{center}}
\newcommand{\ebox}{\end{center}\end{tcolorbox}}
\newcommand{\red}{\colorbox{red}}
\newcommand{\blue}{\colorbox{blue}}
\newcommand{\yellow}{\colorbox{yellow}}
\newcommand{\myit}{\textit}
\newcommand{\mybf}{\textbf}
\newcommand{\li}{\newline}

\graphicspath{{./Images/}}


\geometry{textwidth=16cm, textheight=26cm}

\title{Fiche de lecture : Visualization of software architectures in virtual reality and augmented reality}

\author{Matthys Gaillard}

\date{\today}

\begin{document}
\maketitle
\section{\ul{Pourquoi ce choix ?}}
        \par J'ai choisi ce document\cite{A1}, car il proposait une autre manière de visualiser les larges bases de code. Au lieu de se 
        baser sur la métaphore de la ville, ils proposent de se baser sur une île. De plus, il proposait également de de visualiser les données
        à l'aide de la réalité augmentée.
\section{\ul{Analyse du document}}
\subsection{\ul{Contexte}}
        \par Pour ces chercheurs-ci, l'architecture logicielle est quelque chose d'abstrait. En effet, celle-ci se repose surtout sur du texte écrit.
        Il existe également différents outils donnant la possibilité de mieux l'appréhender, mais ceux-ci ne sont pas toujours très efficaces.
        Pour eux, les gens ont besoin d'une métaphore pour aider à comprendre cette architecture abstraite.
\subsection{\ul{Objectifs}}
        \par L'objectif de leur projet est d'assurer la compréhension de l'architecture logicielle par les utilisateurs. Pour cela, ils ont décidé
        de se baser sur une métaphore différente de celle de la ville. En effet, ils ont décidé de se baser sur une île. Ils ont également décidé
        de l'implémenter à l'aide de la réalité augmentée et de la réalité virtuelle. 
\subsection{\ul{Méthode}}
        \par Pour cela, il collecte les données nécessaire à la visualisation en effectuant de la data mining sur le code source et l'arbre de dépendance.
        Les données collectées sont stockées dans un arbre de database pour effectuer une analyse complémentaire.\li
        \par Ce graphe permettra de générer les îles. Chaque île représente un package. Les îles sont générées en fonction de la taille du package.
        Plus les îles sont proches les unes des autres plus il y a une dépendance entre elle qui est forte.
\subsection{\ul{Résultats}}
        \par Les chercheurs ont ici effectué aucune étude pour savoir si leur méthode était efficace. Cependant, ils ont quand même testé leur application et 
        selon eux, c'était bien plus facile de comprendre l'architecture logicielle et la métaphore de l'île était plus facile à comprendre que celle de la ville surtout en ce qui
        concerne les dépendances entre les différents objets.
\newpage

\bibliographystyle{plain}
\bibliography{biblio}
\end{document}