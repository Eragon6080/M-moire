\documentclass[a4paper,10pt, oneside]{article}
%package

\usepackage[utf8]{inputenc}
\usepackage[T1]{fontenc}
\usepackage[french]{babel}
\usepackage{cite}
\usepackage{url}
\usepackage{pdfpages}
\usepackage{soul}
\usepackage{color}
\usepackage{tcolorbox}
\usepackage{graphicx}
\usepackage{geometry}
 \geometry{textwidth=17cm,textheight=26cm}
\newcommand{\bbox}{\begin{tcolorbox}[colback=red!5!white,
	colframe=red!75!black]\begin{center}}
\newcommand{\ebox}{\end{center}\end{tcolorbox}}
\newcommand{\red}{\colorbox{red}}
\newcommand{\blue}{\colorbox{blue}}
\newcommand{\yellow}{\colorbox{yellow}}
\newcommand{\myit}{\textit}
\newcommand{\mybf}{\textbf}
\newcommand{\li}{\newline}

\graphicspath{{./Images/}}


\geometry{textwidth=16cm, textheight=26cm}

\title{Fiche de lecture : UML-based live programming}

\author{Matthys Gaillard}

\date{\today}

\begin{document}
\maketitle
\section{\ul{Pourquoi ce choix ?}}
    \par J'ai choisi ce document\cite{A1}, car il proposait un environnement entièrement en réalité virtuelle dans lequel tu pouvais à la fois coder et avoir une visualisation
    du code grâce à des diagrammes UML. De plus, à chaque modification et à la sauvegarde du code, le diagramme du code se mettait à jour automatiquement.
\section{\ul{Analyse du document}}
\subsection{\ul{Contexte}}
    \par La compréhension du code et le développement de grandes applications sont des tâches difficiles et coûteuses en temps pour le développeur.
    La visualisation de code a pour but d'aider les développeur à mieux comprendre la structure de leur code, et ainsi de mieux naviguer à l'intérieur de celui-ci.\li
    \par En matière de visualisation logiciel, la réalité virtuelle a déjà fait ses preuves, mais peu de recherches ont été faites en matière de programmation "instantanée".
\subsection{\ul{Objectifs}}
    \par Le but est donc ici de présenter un environnement de programmation en réalité virtuelle, dans lequel le développeur peut coder et visualiser son code en même temps.
    Toutes les étapes sont incorporés directement dans le logiciel, de la création du projet à sa compilation. Une fois, le code compilé, la visualisation se met directement à jour.               
\subsection{\ul{Méthode}}
    \par Dans cette étude, les auteurs ont crée un environnement de programmation entièrement en réalité virtuelle. Cet environnement était divisé en 3 parties : 
    \begin{enumerate}
        \item Le diagramme uml de l'application montrant les interactions
        \item Le code accessible via une tablette virtuelle.
        \item Un jeu illustrant l'application compilée.
    \end{enumerate}
    \par Pour pouvoir tester cet environnement, ils ont fait appel à 20 étudiants en informatique. 16 sont issus du milieu universitaire et les 4 autres de l'industrie.\li 
    \par Pour prouver que leur environnement était efficace, ils ont fait tester le programme compilé (un jeu en 3 étapes) sur 2 environnements différents\footnote{VR IDE et Unity 3D} et puis, il leur ont fait remplir 2 questionnaires\footnote{UEQ et Mann-Whitney}.
\subsection{\ul{Résultats}}
    \par Les résultats montrent que l'environnement virtuel est bien mieux en terme d'utilisabilité et d'expérience utilisateur, mais on garde des problèmes inhérents à la VR. Un environnement classique reste plus approprié en terme de rapidité d'écriture de code.
\newpage 
\bibliographystyle{apalike-fr}
\bibliography{biblio}
\end{document}