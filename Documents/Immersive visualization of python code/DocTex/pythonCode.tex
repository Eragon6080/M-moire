\documentclass[a4paper,10pt, oneside]{article}
%package

\usepackage[utf8]{inputenc}
\usepackage[T1]{fontenc}
\usepackage[french]{babel}
\usepackage{cite}
\usepackage{url}
\usepackage{pdfpages}
\usepackage[french]{babel}
\usepackage{soul}
\usepackage{color}
\usepackage{tcolorbox}
\usepackage{graphicx}
\usepackage{geometry}
 \geometry{textwidth=17cm,textheight=26cm}
\newcommand{\bbox}{\begin{tcolorbox}[colback=red!5!white,
	colframe=red!75!black]\begin{center}}
\newcommand{\ebox}{\end{center}\end{tcolorbox}}
\newcommand{\red}{\colorbox{red}}
\newcommand{\blue}{\colorbox{blue}}
\newcommand{\yellow}{\colorbox{yellow}}
\newcommand{\myit}{\textit}
\newcommand{\mybf}{\textbf}
\newcommand{\li}{\newline}

\graphicspath{{./Images/}}


\geometry{textwidth=16cm, textheight=26cm}

\title{Fiche de lecture : Code City : A comparison of on-screen and virtual reality}

\author{Matthys Gaillard}

\date{\today}

\begin{document}
\maketitle

\section{\ul{Pourquoi ce choix ?}}

        \par J'ai choisi cet article\cite{A3},car le sujet s'éloignait légèrement du sujet de base qui était la compréhension de ligne de code.
        En effet, dans cet article, on parle plus de la compréhension des concepts lié au code dans un langage de programmation\footnote{Python}.
        On reste cependant dans le thème, car l'article propose d'appréhender le langage en utilisant la réalité virtuelle.
        
\section{\ul{Analyse du document}}
\subsection{\ul{Contexte}} 
        \par Les chercheurs mentionné dans l'article, on remarqué qu'apprendre la programmation via Python s'avère être une lutte pour les étudiants en informatique.
        Cependant, la demande de ce genre de profil est en forte croissance depuis quelques années. C'est pourquoi pour eux, il est nécessaire d'adapter les méthodes
        d'apprentissage pour les étudiants. C'est pourquoi, ils ont décidé de tester l'apprentissage de Python via la réalité virtuelle.
\subsection{\ul{Objectifs}}
        \par L'objectif des chercheurs est de créer une application mobile de réalité virtuelle visant à apprendre les compétences basiques de codage en Python.
        Elle a pour but avec une méthode quasi expérimentale de comparer l'efficacité de l'apprentissage de Python via la réalité virtuelle avec une méthode classique d'apprentissage. C'est-à-dire
        un professeur qui explique le code et des exercices à faire.
\subsection{\ul{Méthode}}
        \par Ils ont demandé à 30 étudiants âgés de 18 à 22 ans de participer à l'étude. Ils ont divisé les étudiants en 2 groupes. Un des groupes utiliseront la réalité virtuelle pour apprendre
        tandis que l'autre suivra la méthode d'apprentissage traditionnelle. Le casque de réalité virtuelle utilisé était le google cardboard. De plus, au bout des tests, les élèves sont censés avoir
        appris les mêmes bases du langage de programmation. Comme outils d'évaluation, ils ont utilisé des tests d'utilisabilité, des tests d'hypothèse, des questionnaires classiques\footnote{Questionnaire classique où on répond à des questions}.
\subsection{\ul{Résultats}}
        \par Il s'avère que la méthode employant la réalité virtuelle est perçue comme étant utile et avantageuse pour aider les étudiants à apprendre n'importe où, n'importe quand. Cependant,
        quand l'apprentissage demande davantage de précisions et de détails, mais aussi du contact humain pour répondre à leur question, la réalité virtuelle n'est pas adaptée. En effet, les étudiants ont eu du mal à comprendre les concepts plus complexes.
\newpage

\bibliographystyle{apalike-fr}
\bibliography{biblio}
\end{document}
    
