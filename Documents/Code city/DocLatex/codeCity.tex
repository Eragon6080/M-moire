\documentclass[a4paper,10pt, oneside]{article}
%package

\usepackage[utf8]{inputenc}
\usepackage[T1]{fontenc}
\usepackage[french]{babel}
\usepackage{cite}
\usepackage{url}
\usepackage{pdfpages}
\usepackage[french]{babel}
\usepackage{soul}
\usepackage{color}
\usepackage{tcolorbox}
\usepackage{graphicx}
\usepackage{geometry}
 \geometry{textwidth=17cm,textheight=26cm}
\newcommand{\bbox}{\begin{tcolorbox}[colback=red!5!white,
	colframe=red!75!black]\begin{center}}
\newcommand{\ebox}{\end{center}\end{tcolorbox}}
\newcommand{\red}{\colorbox{red}}
\newcommand{\blue}{\colorbox{blue}}
\newcommand{\yellow}{\colorbox{yellow}}
\newcommand{\myit}{\textit}
\newcommand{\mybf}{\textbf}
\newcommand{\li}{\newline}

\graphicspath{{./Images/}}

\geometry{textwidth=16cm, textheight=26cm}

\title{Fiche de lecture : Code City : A comparison of on-screen and virtual reality}

\author{Matthys Gaillard}

\date{\today}

\begin{document}
\maketitle
\section{\ul{Pourquoi ce choix ?}}
		\par J'ai choisi ce document\cite{A1} parce que son titre amorçait déjà clairement le sujet : une comparaison de la compréhension de la 
		structure d'un code à l'aide d'une visualisation 3D sur un écran classique et un écran de casque de réalité virtuelle. De plus, en lisant l'abstract,
		j'ai pu remarqué que l'étude comparative se fondait sur un logiciel déjà existant : Code City que l'on peut trouver sur internet..
\section{\ul{Analyse du document}}
\subsection{\ul{Contexte}}
		\par Les chercheurs ont remarqué qu'il existait déjà différentes méthodes pour visualiser du code-source. L'ne d'entre-elle est Code City. Ce logiciel
		propose une visualisation en 3D du code sous la forme d'une ville. Chaque morceau d'un projet peut être ainsi représenté par un bâtiment et les caractéristiques
		bâtiment représente des caractéristiques du projet. On peut citer comme exemple, un répertoire est égal à un bâtiment, sa hauteur (la quantité de lignes de code).
\subsection{\ul{Objectifs}}
		\par Les objectifs mentionnés dans cet est de comparer le même code modélisé à l'aide du logiciel Code City sur un écran classique et sur un écran de casque de réalité virtuelle.
		Ils veulent voir s'il y a une différence de compréhension de la structure du code entre les deux méthodes.
\subsection{\ul{Méthode}}
		\par Dans l'article, ils ont mené é expériences avec leur propre implémentation de Code City qui peut à la fois tourné sur un écran classique ou bien sur un écran de casque de réalité virtuelle.
		\par Pour la première expérience, ils ont demandé à 24 participants venant du milieu académique ou industrielle. Á la fin de l'expérience, ils ont demandé un feedback et celui-ci a permis de rectifier le tir 
		pour la deuxième expérience. Celle-ci s'est déroulée avec 26 nouveaux candidats. L'expérience était identiquement mené, mais en prenant en compte les remarques de la première expérience.
\subsection{\ul{Résultats}}
		\par L'expérience a montré que lees participants avec un casque de réalité virtuelle comprenait plus vite la structure du code que ceux avec un écran classique pour atteindre le même niveau de compréhension. 
\end{document}